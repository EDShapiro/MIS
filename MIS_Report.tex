\documentclass{article}
\usepackage{amsmath}
\usepackage{amssymb}
\usepackage{gensymb}
\usepackage{hyperref}
\usepackage{listings}
\usepackage{graphicx}
\usepackage{mhchem}
\usepackage[margin=1in]{geometry}
\linespread{1.0}
\graphicspath{ {c:/Users/EvanD/OneDrive/Pictures/}}
\title{ MIS Thesis Outline}
\begin{document}
\maketitle
\noindent
Consider the following problem
\section{Introduction}
%We are interested in extending work done on the study of plasma size scaling of turbulent heat transport in future nuclear fusion devices.
%Why? 
The history of nuclear fusion research dates to the 1920's, when Arthur Eddington suggested in his book "The Internal Constitution of Stars" that the transmutation of hydrogen atoms through nuclear fusion into helium is responsible for powering the energy released by the sun.\cite{Eddington}. Eddington's hypothesis motivated experimental research with the goal of ceating the conditions necessary for nuclear fusion.
In 1934, Ernest Rutherford and his assistant Mark Oliphant successfully fused 2 deuterium atoms in the following reaction:
\begin{align*}
\ce{^2_1H + ^2_1H ->	^3_2He + ^1_0n + 3.74 MeV}, 
\end{align*}
where the $3.74$ MeV has been converted from nuclear binding energy to kinetic energy,making it an energetically favorable nuclear reaction.\cite{Oliphant_Rutherford}. % their article, "Transmutation Effects Observed with Heavy Hydrogen",\\
This nuclear fusion of deuterium and tritium is even more energetically favorable than D-D fusion, and releases more kinetic energy per nucleon than that of nuclear fission. The D-T fusion chain is safer as an energy source since the fuel is radioactively stable and cannot be weaponized. It is also a environmentally safe or clean energy source relative to fossil fuels. Thus, nuclear fusion is, and has been active area of energy research, with the goal of developing a working fusion reactor that can provide safe and clean electricity.\\
While it nice to be idealistic about the amount of energy that nuclear fusion can provide, designing and engineering a device that can create the conditions necessary to sustain nuclear fusion in a net energy positive manner is quite difficult. There are a narrow range of conditions that guarantee a self-sustaining plasma, and any deviation from this regime requires an external input of heating power to compensate for lost power, and extreme deviations bring into question the feasibility of nuclear fusion as an energy source. Thus it is important to understand the physical mechanisms that can degrade the conditions inside a reactor to be outside of feasible operating conditions. Heat transport out of the plasma via ion transport present a potential way for conditions to be degraded past viable operational conditions. The scaling levels of ion driven heat transport is still not fully understood under the operating conditions of an operational  tokamak reactor, conditions that are outside the range of current experimental fusion reactors. This lack of empirical reference begs he consideration of computer simulations to make heat transport predictions. Gyrokinetic models have been developed that fully characterize ion and electron kinetic behavior and interactions in a fusion environement well enough to model turbulence in a plasma and the resulting heat transport, and, with the power of modern numerical simulations and parallel computer systems, current fusion simulations can predict transport at future tokamak reactor scales. However, the parameter space of tokamak reactors is not fixed but rather reside in a range condiitons, thus it is important predict transport in this full range of conditions. Due to the computational expense of simulating operational conditions of a tokamak reactor, only a handful of uncertainty propogation experiments have been performed in the described conditions. We thus seek to outline and use methods that allows us to propogate uncertainty through a fusion plasma model, or an approximate model, in a mathematically rigorous manner to predict heat transport in a working fusion reactor under uncertain conditions.\\
\newpage
\section{background}
To understand why nuclear fusion is such an energy rich reaction we must first develop a physical model for the reaction. Nuclear fusion occurs when the nuclei of two elements combine due to the nuclear force exerted force, producing a heavier element, converting potenital energy into kinetic energy, and potentially releasing subatomic particle. The nuclear fusion reaction that is of primary interest in fusion energy research is the deuterium($\ce{^2_1H}$)-tritium$\ce{^2_1H}$ reaction have enough energy to overcome the Coulomb repulsion force, and get close enough together so that their nucleii fuse together. This results ini the formation of an $\alpha$ particle, the release of a neutron, and the conversion of nuclear potential energy to kinetic energy. This release of kinetic is due to the decrease of potential energy in the nucleus as nucleons move closer to each other, thereby increasing the energy binding the nucleus together. The increase of kinetic energy during a fusion event is what motivates nuclear fusion research, as, at large enough scales, this process yields enough energy to be leveraged to power a steam turbine to produce electricity, potentially providing a clean, safe source of energy. Progress has been made in creating nuclear fusion reactors that can sustain a fusion plasma, one of the major issues of particular interest in this project is extending work that has been on understanding plasma size scaling of turbulent heat transport in future nuclear fusion devices. [1]\\
\newpage
We are interested in nuclear fusion events that maximize likelihood of occurence and energy ouput after fusion. Considering these two factors, the reaction of primary interest in fusion research is the deuterium-tritium fusion reaction:  
\begin{align*}
\ce{^2_1H + ^3_1H ->	^4_2He + ^1_0n}, 
\end{align*}	
where $\ce{^2_1H}$ is deuterium (D), $\ce{3_1H}$ is tritium (T),  $\ce{^4_2He}$ is an $\alpha$ particle, and $\ce{^1_0n}$ is a neutron.
Taking into account the binding energy of each element, the mass of the system before and after this nuclear fusion event is given by:
\begin{align*}
\ce{(2 - 0.000994)m_p +(3 - 0.006284) m_p -> (4 - 0.0427404)m_p + (1 + 0.001378)m_p}
\end{align*}
where $m_p=1.6726*10^{-27}kg$ is the proton mass. The difference in mass before and after the reaction is thus $\delta_m = -0.01875m_p$.\\
Using Einstein's famous equation $\mathcal{E} = mc^2$, where $c$ is the speed of light, we find that energy-mass conversion to be
\begin{align*}
\mathcal{E} = \delta_m *c^2 = 0.01875*1.6726*10^{-27}kg c^2 = 17.9 MeV,
\end{align*}
where the mass has been converted to kinetic energy.
A further calculation shows that $1/5th$ of this energy is distributed to the $\alpha$ particle, and the other $4/5$th is distributed to the neutron. In a fusion reactor these two particles serve two important, and different, functions. When the number of fusion events is scaled to a large enough level, the resulting number of high energy neutrons can be leveraged to power a steam turbine, which through the inductive process, produces electricity. The high kinetic energy of the $\alpha$ particles are used to increase the the kinetic energy of supplied D-T fuel, reducing the energy required for continuous external heating. creating a self-sustaining fusion process. \\
Since both deuterium and tritium have a net positive charge there is a repelling Coulomb force between them that prevents fusion from occuring in low temperature environments. Additionally in low temperature environments the cross sectional area of the D-T mixture is too low to make fusion an energy positive process. To overcome the Coulomb barrier, and increase the effective cross-section, the deuterium and tritium must be supplied enough kinetic energy to surpass the potential energy barrier to get within a particular radial distance from each other. "A positive energy balance is possible if the fuel particles can be made to interact to lose their energy. To achieve this the particles must retain their energy and remain in the reacting region for a sufficient time." This leads to a necessary confinement time of a sufficient number of particles within a particular region, or number density $n$, to create the conditions necessary for sustained fusion. The confinement time, $\tau_E$, is the primary parameter of interest in this project, as we are studying diffusive processes that degrade the experimental confinement time below the confinement time necessary for sustained fusion.\\
 In equillibrium, the kinetic energy of the D-T fuel mixture in a fusion reaction follows a Maxwellian distribution, and the high energy particles in the upper tail of the distribution are the source of the fusion reactions. The necessary average temperature to achieve desired fusion rates in this tail is 10keV, or around 100 million degrees centigrade. [1] The high temperature environment leads to complete ionization of the D-T mixture, yielding an equal number of electrons $n_e$ and ions $n_i$. The net equality of charge leads to an electrically neutral system in equillibrium, which is highly responsive to external electric and magnetic fields; a gas in this state is referred to as a plasma.\\
We would like to keep all of the fusion particles confined inside of the reactor, however the high energy and temperature of the plasma particles preclude using the physical wall of the reactor for confinement.[1] There are two important factors that must be considered when designinng a device to confine a fusion plasma: the plasma state of the plasma the fact that we want to prevent the high energy particles of the plasma from colliding with the material walls of the device. Through the Lorentz force, the ions and electrons of the plasma can be confined with strong magnetic fields interior to the fusion device. A tokamak reactor is a toroidal nuclear fusion device that generates helical magnetic fields that provide a confining path for the ions and electrons of the plasma. The path traversed by the ions and electrons prior to reaching a material wall is a million times the length of the reactor.\\
\newline
\section{Power Balance}
For a fusion reactor to be viable from both an energy and economic persepective it must meet power balance condition: $P_{out} >> P_{in}$, where $P_{out}$ is the power retrieved from the high energy neutrons, and $P_{in}$ is the power necessary to sustain these fusion process while the reactor is in operation. [2] If the power necessary to sustain the fusion process is of the same order as the power recovered from the high energy neutrons, then our process will not be a worthy investment. Thus, understanding the input power $P_{in}$ required to sustain nuclear fusion in a working scale tokamak reactor is of critical importance. Predicting $P_{in}$ requires that we account for all of internal fusion plasma power sources, $P_{source}$, and powers sinks, $P_{sink}$, which we will do next. Once a full accounting of the sources and sinks is completed, the conditions necessary for a self-sustained fusion reaction can be determined. \\

The primary internal fusion power source are the $\alpha$-particles that are created during the fusion reaction. The magnetic fields generated by the fusion reactor confines the positively charged $\alpha$-particles. Under confinement, these $\alpha$-particles transmit the 3.5MeV they obtained during the fusion reaction to incoming D-T fuel, imparting enough kinetic energy for D-T fusion to occur. Taking into account $\alpha$-particle velocity, cross-sectional area as a function of velocity $\sigma(v)$, total ion density $n$, and energy per $\alpha$-particle, $E_{\alpha}$, the $\alpha$-particles heating or power per unit volume is
\begin{align*}
p_{\alpha} = \frac{1}{4} n^2 <\sigma v> E_{\alpha},
\end{align*}
with $<\sigma v>$ being the fusion reaction rate assuming a Maxwellian distribution function. 
Integrating over the entire volume yields the total $\alpha$-particle power in the plasma.
\begin{align*}
P_\alpha &= \int_V \frac{1}{4} n^2 <\sigma v> E_{\alpha} d^3x\\
	     &=  \frac{1}{4}\overline{n^2<\sigma v>}E_{\alpha}V.
\end{align*}
 with $V$ being the volume of the reactor.\\
Now we account for power losses in the plasma. Due to temperature gradients and Coulomb collisions, in a tokamak reactor power is continuously leaving the plasma as ions and electrons move from the plasma to the reactor material walls. This lost power should ideally be completely replaced by $\alpha$-particle heating, and if it is not then external heating must be used to replace the lost power. "The average energy of plasma particles at a temperature $T$ is $\frac{3}{2} T$..." [1]. The electrons and ions have the same temperaturee and quantitity, so if we let the ion number density per unit volume be $n$, the kinetic energy density of the  system is $3nT$. Integrating over the entire volume yields the following total plasma energy 
\begin{align*}
W &= \int_{V}3nTd^3x\\
    &=3\overline{nT}V.
\end{align*}
We define the characteristic time $\tau_E$ to be the amount of time that heating power, once generated, stays in the plasma. This parameter is very important in nuclear fusion, as it has a minimum threshold that guarantees ignition, or self-sustaining fusion in a fusion reactor.\\ The amount of heating power that leaves is given by 
\begin{align*}
P_H = \frac{W}{\tau_E}.
\end{align*}
The yields the following power balance equation
\begin{align*}
P_H + P_{\alpha} &= P_L\\
P_H + \frac{1}{4}\overline{n^2 <\sigma v>}E_{\alpha} V,
\end{align*}
where $P_H$ is the applied heating power.\\
We now want to consider the power in a fusion reactor after fusion has been initiated, and determine the conditions necessary for self sustaining fusion. For self sustained fusion,  there must be enough $\alpha$-particle heating power to sustain the fusion process, and negate the need for external heating. Thus, in the power balance equation, this means letting $P_H =0$ and rearranging the power balance equation to arrive at the following inequality
\begin{align*}
n\tau_E > \frac{12}{<\sigma v>}\frac{T}{E_\alpha}. 
\end{align*}
Considering an operating temperature range of 10-20 $keV$, $<\sigma v> = 1.1 \times 10^{-24} T^2 m^3 s^{-1}$, we arrive at the following condition for a self-sustaining plasma
\begin{align*}
nT \tau_E > 5 \times 10^{21} m^{-3} keV s
\end{align*} 
in which the density and temperature profile inside the plasma are parabolic, and $n,T$ are the peak values of the profile. Given an operating temperature of $10$ keV, and a number density of $10^20 m^{-3}$ yields a time constant of $\tau_E =5$ s, or the required energy confinement time in a tokamak reactor is about 5 s. This implies that, if all the power lost from the system is to be balanced by $\alpha$ particle heating, that the characteristic time scale of heat transport should be on the order of 5 seconds. Thus, we need to fully understand all of the physical transport mechanisms inside of the fusion reactor that degrade these conditions, and \\
Confining energy in the plasma for a time $\tau_E$ means that conditions must exist in the reactor that constrain heat from flowing at a rate faster than that dictated by the ignition time. This is one of the major problems being adressed in current nuclear fusion physics: predicting the rate at which heat leaves a fusion plasma under operational conditions in an ITER scale reactors


\section{Important Plasma Definitions \& Behavior}

To be self-consistent in this paper, we establish some important plasma definitions, different types of plasma motion, and the plasma confinement strategies employed in a tokamak fusion reactor. 
First
In most plasma models and simulations, we consider the ions to be fixed, as they are much heavier than electrons, and do not react to field fluctuations as quickly as electrons. Thus most plasma fluc


\textbf{General Behavior of Charged Particles in an Electric and Magnetic Fields.}
This section is dedicated to summarizing important plasma definitions that are either implied or used in the rest of this project:
Debye Shielding\\
Plasma Frequency\\
Group Behavior\\
\section{Magnetic Confinement}

\section{Traveling Waves \& Fluctuations}

\subsection{H-Mode Operation vs L-Mode Operation}

In the 80s it was found that above certain power levels the plasma transitions from a low energy mode (L-Mode) to a high energy mode (H-Mode), and in this high energy mode the confinement time of the plasma is reached. However, instabilities in the plasma edge, or edge localized modes (ELM) can cause the the plasma to crash from the H-mode back down to the L-mode. These edge localized modes are caused by turbulence driven transport, induced by ion temperature gradients in the plasma.  flow must fully characterize all of the transport processes driving heat loss. A full characterization of the heat transport processes is beyond the scope of this master's report, but each of the transport mechanisms will be summarized here.

\section{Plasma Diffusion \& Transport}

Drift phenomenon.
Magnetic Confinement Principles.
Traveling waves in a plasma.
Kinetic Theory
Heating Model.
Heat transport - Classical, Neoclassical, and Turbulent Transport.
$\rho*$ scaling.


\subsection{Transport Scaling and Scaling Laws}
Predicting plasma behavior in fusion reactors is made possible through the use of dimensionless groups of parameters 

\section{Plasma Models}
Description of Models
\subsection{Fluid Models}
\subsection{Kinetic Models - Vlasov-? }
To accurately predict ion temperature gradient driven turbulent transport in a fusion interparticle collisions must be taken into consideration  


\section{Literature Review}
$rho*$ parameter can be scaled to reflect larger plasma size than current reactors allow

Predicting the scaling of turbulence driven heat transport in a fusion plasma at future tokamak sized reactors is critical to predicting performance, and to ensure that reactors the nececessary confinement properties for sustained fusion. Thus, scaling studies are of critical importance in fusion reactor design studies. As mentioned previously, the net heat transport must be balanced by auxiliary heating power, so understanding heat transport scaling is vital from an economic perspective. Additionally, stability of the plasma and controlling turbulence is critical to consistent tokamak performance, and edge model simulations may be employed in the future when simulating tokamak performance to ensure the reactor is within the desired operating regimes. \\

Experimental transport research has focused on ion or electron transport scaling with the dimensionless parameter $\rho*$, as ions and electrons are the primary channels for heat transport out of the plasma. Experiments performed in current fusion reactors have indicated that in H-Mode operation that the scaling of ion driven transport is much larger than electron driven turbulent transport, making the ion channel the the primary channel studied in simulation studies. In H-Mode operation the ion transport scaled linearly (Bohm-like) with plasma size, while electron transport remained constant (gyroBohm-like) as plasma size was scaled. Thus, the focus of most simulation studies is on ion temperature gradient driven turbulent transport. \\

The scaling of heat transport derived in current experimental reactors cannot necessarily be extrapolated to future operational reactors, as, in current reactors, the dimensionless plasma radius $\rho*$ cannot be scaled to future operational values. However, with the development of parallel computing, modern gyrokinetic theory, and advanced numerical simulations and algorithms, it has become possible to accurately simulate turbulence driven heat transport from first physical principles.

Previous $\delta_f$ simulation studies involved fixing the ion temperature gradient, $R/L{ti}$ and scaling the plasma size via the dimensionless parameter $\rho*^{-1} = a/\rho_{ti}$, where $a$ is the radial length of the tokamak reactor, and $\rho_{ti}$ is the thermal ion Larmor radius, $R$ is major radius of the tokamak, and $L_{ti}$ is the ion temperature length scale, and then observing the normalized ion heat flux, $q_i/(\chi_{GB}n_iT_i/a)$, at a particular radial distance, $r = 0.5a$ with $q_i $ being the ion heat flux, $\chi_{GB}$ is the Bohm diffusion constant, $n_i= 10^{19}m^{-3}$, and $T_i$ is the ion temperature.c


In a working fusion reactor, auxiliary heating methods will be used to balance the power lost from the plasma via heat transport. Depending on how transport scales with reactor size, different heating power scaling and strategies will need to be employed: in the case of Bohm like scaling of transport with reactor size, the auxiliary heating will need to increase linearly with reactor size, while gyro-Bohm scaling means that auxiliary be held constant. This difference is critical, as any additional input power affects the global confinement properties of the plasma, and increases heat transport. Additionally, the input power that will be generated by $\alpha$ particle heating in future tokamak reactors is significantly larger than current fusion experiments, and so predicting how $\alpha$ particle heating affects transport scaling is impossible with current experimental reactors.  Thus, to accurately model heat transport, it is necessary to incorporate a heating model that captures the input power $P_in$ from auxiliary heating and $\alpha$ particle heating.\\

In Nakata and Idomura it was shown that, due to ion avalanche events in the plasma, the net ion heat flux "increased rapidly above the nonlinear critical gradient," yielding reduced confinement properties as heating power is scaled. This increase in ion heat flux indicates that maintaining the necessary temperature gradient for H-Mode operation becomes more difficult as auxiliary power increases, which is problematic since that is the operating mode that future reactors will work in. or that the input power degrades the confinement property $\tau_i$. In the same paper,the $\rho*$ scan demonstrated that the PDF for $q_i$ with $\rho* = 600$, and without a heating power model, matched the $q_i$ at $(\rho* = 300, P_in = 8MW)$, which suggest that ion transport scaling in the simulated environment is predominantly driven by input power and not plasma size. These results demonstrate important of fully characterizing the role that input power, and input power scaling, play in ion driven heat transport and scaling in a fusion reactor. \textbf{Figures from Nakata Idomura 2014 here}\\
Under operation there will be uncertainty in the auxiliary heating input parameters, as controlling radial heat deposition via nuetral beam injection or externally generate eletric fields is inexact. Thus, it necessary to consider the full range of possible radial heating profile scenarios to accurately predict the range of heat transport levels at prescribed power input levels. In this project/thesis we build on previous work done in transport scaling research by work performed in [NAKATA, IDOMURA, and other heating power scaling experiments] by incorporating the uncertainty that exists in the heat deposition methods into the heating model used in the transport scaling experiments discussed above. Propagating the uncertainty through the heating model will allow us to develop a PDFs of $q_i$ as the radial deposition profile is varied, allowing us to determine the sensitivity in heat flux to the initial deposition profile.\\
\section{Heating Sources}
Per [Y. Idomura1, H. Urano2, N. Aiba2 and S. Tokuda2], "A source term requires and empirical modeling." This on-axis radial heating model with a fixed power input provides the energy for momentum and heat transport driven by ITG turbulence. [[Y. Idomura1, H. Urano2, N. Aiba2 and S. Tokuda2] The sink model fixes the temperature $T_i$ and parallel velocity $U_{||}$  at the pedestal boundary to reflect that of H-Mode boundary conditions.\\
The equations of the model are given by
\begin{align*}
S_{src} &= \nu_h A_{src}(r)(f_{m1} - f_{m2})\\
S_{snk} &= \nu_s A_{snk}(r)(f-f_0)\\	
\end{align*}
where $A_{src}(r)$ is the source deposition profile of radial on-axis heating, and $A_{snk}$ is the heat absorption deposition profile. The heating and cooling rates are given by $\nu_h$ and $\nu_s$, $f_{M1}, f_{M2}$ are two shifted Maxwellian distribution functions with different temperatures, and $f$ is the ion and electron dsitribution, and $f_0$ is the initial distribution. profile$f_{M1}$, $f_{M2}$, are chosen to impose fixed power input $P_in$ with no momentum or particle input.  The sink equation maintains the initial plasma boundary velocity and temperature using a Krook operator. Possible equations enforcing this condition?\\

\vspace{0.01cm}

\noindent
\subsection{Defining the Quantity of Interest}
We now consider forward propogation of the uncertainty associated with the physical parameters through the 5-D gyrokinetic ion model, and adiabatic electron model. As mentioned, this involves solving a system of partial differential equations that describe the ion and electron distribution functions in a 5-D phase space, which have conservation constrains enforced via various equations, and which are couple to Maxwell's systems of equations to propogate the charged particles through phase space. The solution of these system in the regime we are interested in, with ion and electron density on the order of $10^{19}m^{-3}$, takes NERSC parallel computers on the order of hours to days of computational time.\\


\vspace{0.01cm}
\noindent

For the sake of clarity and compactness in writing, we simplify the full 5-D gyrokinetic simulation and heating model to be a black-box mapping, $f$, which maps the parameter space, $\mathbb{P}$, where $\mathbb{P} \subset \mathbb{R}^4$ are the inputs of the heating model, to a space containing our quantity of interest  (QOI), $Q\subset \mathbb{R}$, or
(I need to refine the domain and co-domain definition for the paper.)
\begin{align*}
f:\mathbb{P} \to Q.
\end{align*}
We assume that our parameters are either normally distributed $\mathcal{N}(\mu, \sigma)$, or uniformly distributed $\mathcal{U}(0,1)$. and we are interested in predicting the expected value of the normalized net heat flux, $\hat{q}_i = q_i/(\chi_{GB}n_iT_i/a)$, as we vary over the entire parameter space, $\mathbb{P}^4$. Thus, we define the normalized heat flux to be a function of the parameter space, or $\hat{q}_i = \hat{q}_i(\textbf{p})$, where \textbf{p} $\in \mathbb{P}^4$, as well as a normalized heat flux distribution function $f(\hat{q}_i(\textbf{p}))$ over the entire parameter space. The expectation value is then
\begin{align*}
\mathbb{E}(\hat{q}_i(\textbf{p})) = \int_{\mathbb{P}^4}\hat{q}_i(\textbf{p}) f(\hat{q}_i(\textbf{p}))d\textbf{p}.
\end{align*}
\subsection{Methods for Determining $\mathbb{E}(\hat{q}_i(\textbf{p}))$}
Since the input parameter space is continuous it is impossible to evaluate this integral using traditional quadrature methods, as the number of points necessary to evaluate the integral exactly grows with the degree of the parameter space, resulting in a number of simulation runs that exceeds our computational budget.  (Find exact quadrature error.) Thus, in section we consider alternative numerical methods for evaluating our QOI.
\subsubsection{Monte Carlo Integration}
Monte Carlo Integration provides a way to evaluate this integral via random sampling of the parameter space $\mathbb{P}$ 	and, by t he law of large numbers, converges to the true solution
 Randomly sampling from the parameter distribuion space provides a way to approximated 
Monte Carlo sampling provides a way reduce the uncertainty in our QOI, but the reduction in uncertainty comes at a great cost, as the number of samples $N$ required to reduce the uncertainty in the QOI scales exponentially with the dimension, $d$, of the parameter space. or


\section{UQ Model Problem - Landau Damping}
\subsection{Physical Background - Landau Damping}
To motivate the uncertainty quantification problem we are solving in a high fidelity setting, and to demonstrate the methods for sparse grid inerpolant model surrogates, we introduce a toy model to perform uncertainty quantification with, Landau damping. Lev Landau is responsible for properly deriving the dispersion relation for electron plasma oscillations by using contour integration to solve the Vlasov equation in the complex field. What was discovered is that, given a traveling electric field fluctuation in a plasma, electrons traveling with a velocity \textbf{v} close to the phase velocity, \textbf{v}$_{\phi}$, of the electric field can exchange energy with the field. An electron with \textbf{v}  $\approx$ \textbf{ v}$_{\phi}$, moves opposite the field, in the field reference frame, and can interact with the field. An electron moving slower than the field will be accelerated when it reaches the peak of the wave in the direction that the wave is traveling, increasing kinetic energy of the electron and reducing the energy of the field. Electrons traveling faster than the phase velocity impart energy to the wave, decreasing their kinetic energy. Assuming the electron velocities have a Maxwell distribution, there are more slow electrons than there are fast electrons, thus there is an overall damping of the field. This discovery was monumental, as it demonstrated the occurrence of collisionless energy exchange in a plasma, and that energy can be deposited into a plasma using electric fields.\\

\vspace{0.01cm}

There are two types of Landau damping that we consider: linear and nonlinear Landau damping. If the field potential, $\phi$, is small then the electric field amplitude does not effect the kinetic energy of the electron, and the field and velocity equations can be linearized. If $\phi$is larger than the kinetic energy of the electron, then the electrons can become trapped in the potential of the electric field. When trapped, electrons completely change directions relative to the wave when reflected off of the potential, thus the velocities of the electron are greatly affected by the oscillating field, and the distirbution function $f(v)$ is greatly disturbed, no longer allowing for linearization of our system of equations. \\

\vspace{0.01cm}

In our simulation and UQ example we consider both the linear and nonlinear Landau damping examples\\ 

\vspace{0.01cm}
\subsection{Vlasov-Boltzmann Description}
To setup the problem, we consider the most general domain of a 6-D phase space $\Omega = \Omega_x\times \Omega_v = [0,L]\times[0,L]\times[0,L]\times \mathbb{R}^3$. To properly treat Landau damping, kinetic theory is employed to model how the electrons in the distribution close to the electric field phase velocity interact with electric field. The Vlasov-Boltzmann equation is the standard equation used to model the electric field fluctuations:

\begin{align*}
\frac{\partial f}{\partial t} + \textbf{v}\cdot \nabla f + \frac{q}{m}(\textbf{E} + \textbf{v}\times \textbf{B})\cdot \frac{\partial f}{\partial \textbf{v}}=0,
\end{align*}
along with divergence free magnetic field $\nabla \cdot$\textbf{B}$= 0$.\\
Since the ions are massive in size relative to the electrons it is assumed that they do not react to fluctuations in the electric field, so we treat the ion density and related quantities as fixed, which yields $\rho_{ion} = 1$.\\
\vspace{0.01cm}
Coupling the Vlasov-Boltzmann system to Poisson's equation allows us to solve for quantities of interest such as the potential, charge density, current, and electric field by evaluating the first moment of the electron distribution function: 
\begin{align*}
\rho_{e} &= -\int f d^3v\\
-\nabla^2 \Phi &= \rho_{ion} + \rho_{e}\\
E&:= - \nabla \Phi.
\end{align*}
The kinetic energy of the electrons is calculated by taking the second moment of distribution function, and the electric field energy is calculated with the classical energy equation $\mathcal{H}_E(t) = \frac{1}{2}\int_{\Omega_x} |E|^2 d^3x$, yielding a total time dependent  Hamiltonian
\begin{align*}
\mathcal{H}(t) = \frac{1}{2}\int_{\Omega} f v^2 dx^3dv^3 +  \frac{1}{2}\int_{\Omega_x} |E|^2 d^3x.
\end{align*}
The Vlasoz-Boltzmann equation is a homogeneous Cauchy equation, and therefore a conservative system, derived by taking the total time derivative of the distribution function $f(\textbf{X}(t),\textbf{V}(t),t)$
\begin{align*}
\frac{D f}{Dt} = \frac{\partial f}{\partial t} + \frac{d \textbf{X}}{dt}\frac{\partial f}{\partial \textbf{X}} + \frac{d\textbf{V}}{d t} \frac{\partial f}{\partial \textbf{V}} =0,
\end{align*}
which means that volumetric elements of the distribution stay contant along trajectories in phase space, or
\begin{align*}
f(X(t), V(t), t) =  f(X(0), V(0), 0) \qquad \forall t\ge 0.
\end{align*}
The Poisson equation along with the method of characteristics is used to solve this system, with the characteristics evolving according to the following equations of motion [Ameres citation]
\begin{align*}
\frac{d X(t)}{dt} &= V(t)\\
\frac{d V(t)}{dt} & = -(E(t,X(t)) + V(t)\times B(X(t),t)).
\end{align*}
\vspace{0.01cm}
\noindent
We now consider a 1-D electric field oscillation example that demonstrates Landau damping, with an additional small constant background electric field. We consider a uniform plasma with background distribution $f_{0}($\textbf{v}) is assumed, and \textbf{B}$_0=B_0$(\textbf{r}) \textbf{E}$_0= E_0$(\textbf{r},t), with  \textbf{B}$_0$ and \textbf{E}$_0$ being the background fields. 
is introduced in the plasma, causing electron oscillations. This yields the perturbed distribution function $f($\textbf{x,v},t), with a distribution perturbation $f_1($\textbf{x,v},t), 

\begin{align*}
f(\textbf{r,v},t) = f_0(\textbf{v}) + f_1(\textbf{r,v},t).
\end{align*}

\noindent
If we assume a traveling electric field perturbation, \textbf{E}$_1$, then the total electric field is given by
\begin{align*}
\textbf{E}(\textbf{r},t) = \textbf{E}_0(\textbf{r},t) + \textbf{E}_{1}(\textbf{r},t).
\end{align*}
Now we reduce this system of equations by assuming a 1-D perturbation in the $x$-coordinate. This simplifcation still allows us to capture the essential physics under consideration.\\

The distribution function reduces to
\begin{align*}
f(x, v_x,t) = f_0(x,v_x,t) + f_1(x,v_x,t)
\end{align*}s
Assume an electric field only in the $x$-direction, so \textbf{E}$_0$ = $E_{0,x}(x,t)$, and assume the following form of the perturbed electric field

\begin{align*}
\textbf{E$_1$} = E_1e^{i(kx-\omega t)},
\end{align*}
then total electric field is given by 
\begin{align*}
E_x(t) = E_{0,x}(x,t) + E_1e^{i(kx-\omega t)}.
\end{align*}
Since the electric field is 1-D,

 $$f_1 \propto e^{(i(kx - \omega t)}.$$

 We also assume that, since the ions are massive, and essentially not moving on the timescales of interest, that we can ignore their behavior. This means we can solve a single Vlasov-Boltzmann equation for the plasma electrons. This model can be reduced even further by removing the background electric field, and will still retain the Landau damping behavior.\\
Substituting $E_x(x,t)$, , and $f(x,v_x,t)$ into the Vlasov-Boltzmann equation we arrive at the:
\begin{align*}
\frac{\partial(f_0(\textbf{v}) + f_1(x,,v_x,t))}{\partial t} +  v_{x}\frac{\partial( f_0(\textbf{v}) + f_(x,v_x,t))}{\partial x} + (E_{0,x} + E_1e^{(i(k_x x-\omega t)})\frac{\partial(f_0(\textbf{v}) + f_1(x,v_x,t))}{\partial v_x}=0,
\end{align*}
which reduces to
\begin{align*}
-i\omega f_1 +  v_{x}ik_xf_1 + (E_{0,x} + E_{1,x}e^{(i(k_x x-\omega t)})\frac{\partial( f_0 + f_1)}{\partial v_x} =0,
\end{align*}
when considering the time and spatial independence of $f_0$.\\
In our model we consider $E_{0,x}$ to be a constant background electric field that, over time, alters the distribution function $f(v)$.\\
Finally, Poisson's equation yields
\begin{align*}
\epsilon_{0} \nabla \cdot \textbf{E}_1 &= ik \epsilon_0E_x\\
& = -en_1\\
& = -e \int \int \int f_1 d^3 v,
\end{align*}
which closes our system of equations.
In the linear regime of Landau damping the term $E_{1,x}e^{(i(k_x x-\omega t)}\frac{\partial(  f_1)}{\partial v_x}=0$, as this is a second order perturbation. However, when considering electron trapping in the electric field, the distribution $f(v)$ is greatly disturbed around the velocity $v=\omega/k_x$, so the second order perturbation quantity cannot be ignored[cite chen].\\
\subsection{Simulation Methodology and Outline}
We wish to simulate the above model . To do so we use a symplectic Runga-Kutte method [cite Ameres]


We are concerned with developing an accurate distribution of $q_i$ as we propogate the uncertainty in each parameter through the black box. We first address the data sampling method that will be used in our experiment. a

There are various ways to go about doing this: the basic workhorse for error reduction is Monte Carlo sampling, which reduces error as $\frac{1}{\sqrt{n}}$, where $n$ is the number of samples. Due to the restrictions noted above, and limited computational resources, we are restricted in the number of samples that we can propogate through the map, making Monte Carlo sampling infeasible in this project. So we exist in a scenario with sparse data $\mathbb{R}^4$ that will maximize the amount of information we yield $Q$.  One method that is well understood is multi-variable polynomial interpolation or regression on our data to approximate the full mapping. We thus yield a finite number of points


We now transition to the uncertainty quantification setting, to discuss



We are doing forward propogation on the heating model which is of critical importance  in predicting





\subsection{Sparse Grid Surrogate Model}
The 
What is a surrogate model? How can surrogate models be helpful when performing  Hi-Fi plasma simulations. Examples. 
How do we develop a surrogate model? Be very clear here.


\subsection{UQ as a way to build a PDF on QOI from a plasma simulation}
Full description of sources of uncertainty in simulation - I.E. uncertainty of input parameters, variables\\
Description of propagation of error through a forward non-linear map. Give basic example here using the nonlinear Landau damping UQ example.\\

\section{Description of Proposed Work}

\section{Conclusions}
\end{document}