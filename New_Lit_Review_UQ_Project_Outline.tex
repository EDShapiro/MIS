\documentclass{article}
\usepackage{amsmath}
\usepackage{amssymb}
\usepackage{gensymb}
\usepackage{hyperref}
\usepackage{listings}
\usepackage{graphicx}
\usepackage{mhchem}
\usepackage[margin=1.25in]{geometry}
\linespread{1.0}
\graphicspath{ {c:/Users/EvanD/OneDrive/Pictures/}}
\title{ MIS Thesis Outline}
\begin{document}
$rho*$ parameter can be scaled to reflect larger plasma size than current reactors allow

Predicting the scaling of turbulence driven heat transport in a fusion plasma at future tokamak sized reactors is critical to predicting performance, and to ensure that reactors the nececessary confinement properties for sustained fusion. Thus, scaling studies are of critical importance in fusion reactor design studies. As mentioned previously, the net heat transport must be balanced by auxiliary heating power, so understanding heat transport scaling is vital from an economic perspective. Additionally, stability of the plasma and controlling turbulence is critical to consistent tokamak performance, and edge model simulations may be employed in the future when simulating tokamak performance to ensure the reactor is within the desired operating regimes. \\

Experimental transport research has focused on ion or electron transport scaling with the dimensionless parameter $\rho*$, as ions and electrons are the primary channels for heat transport out of the plasma. Experiments performed in current fusion reactors have indicated that in H-Mode operation that the scaling of ion driven transport is much larger than electron driven turbulent transport, making the ion channel the the primary channel studied in simulation studies. In H-Mode operation the ion transport scaled linearly (Bohm-like) with plasma size, while electron transport remained constant (gyroBohm-like) as plasma size was scaled. Thus, the focus of most simulation studies is on ion temperature gradient driven turbulent transport. \\

The scaling of heat transport derived in current experimental reactors cannot necessarily be extrapolated to future operational reactors, as, in current reactors, the dimensionless plasma radius $\rho*$ cannot be scaled to future operational values. However, with the development of parallel computing, modern gyrokinetic theory, and advanced numerical simulations and algorithms, it has become possible to accurately simulate turbulence driven heat transport from first physical principles.

Previous $\delta_f$ simulation studies involved fixing the ion temperature gradient, $R/L{ti}$ and scaling the plasma size via the dimensionless parameter $\rho*^{-1} = a/\rho_{ti}$, where $a$ is the radial length of the tokamak reactor, and $\rho_{ti}$ is the thermal ion Larmor radius, $R$ is major radius of the tokamak, and $L_{ti}$ is the ion temperature length scale, and then observing the normalized ion heat flux, $q_i/(\chi_{GB}n_iT_i/a)$, at a particular radial distance, $r = 0.5a$ with $q_i $ being the ion heat flux, $\chi_{GB}$ is the Bohm diffusion constant, $n_i= 10^{19}m^{-3}$, and $T_i$ is the ion temperature.c


In a working fusion reactor, auxiliary heating methods will be used to balance the power lost from the plasma via heat transport. Depending on how transport scales with reactor size, different heating power scaling and strategies will need to be employed: in the case of Bohm like scaling of transport with reactor size, the auxiliary heating will need to increase linearly with reactor size, while gyro-Bohm scaling means that auxiliary be held constant. This difference is critical, as any additional input power affects the global confinement properties of the plasma, and increases heat transport. Additionally, the input power that will be generated by $\alpha$ particle heating in future tokamak reactors is significantly larger than current fusion experiments, and so predicting how $\alpha$ particle heating affects transport scaling is impossible with current experimental reactors.  Thus, to accurately model heat transport, it is necessary to incorporate a heating model that captures the input power $P_in$ from auxiliary heating and $\alpha$ particle heating.\\

In Nakata and Idomura it was shown that, due to ion avalanche events in the plasma, the net ion heat flux "increased rapidly above the nonlinear critical gradient," yielding reduced confinement properties as heating power is scaled. This increase in ion heat flux means indicates that maintaining the necessary temperature gradient for H-Mode operation becomes more difficult as auxiliary power increases, which is problematic since that is the operating mode that future reactors will work in. or that the input power degrades the confinement property $\tau_i$. In the same paper,the $\rho*$ scan demonstrated that the PDF for $q_i$ with $\rho* = 600$, and without a heating power model, matched the $q_i$ at $(\rho* = 300, P_in = 8MW)$, which suggest that ion transport scaling in the simulated environment is predominantly driven by input power and not plasma size. These results demonstrate important of fully characterizing the role that input power, and input power scaling, play in ion driven heat transport and scaling in a fusion reactor.\\
Under operation there will be uncertainty in the auxiliary heating input parameters, as controlling radial heat deposition via the heat injection methods used is inexact. Thus, it necessary to consider the full range of possible radial heating profile scenarios to accurately predict the range of heat transport levels at prescribed power input levels. In this project/thesis we build on previous work done in transport scaling research by work performed in [NAKATA, IDOMURA, and other heating power scaling experiments] by incorporating the uncertainty that exists in the heat deposition methods into the heating model used in the transport scaling experiments discussed above.

We are doing forward propogation on the heating model which is of critical importance  in predicting

\end{document}